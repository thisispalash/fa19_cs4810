\documentclass[11pt,letterpaper]{article}
\usepackage{preamble}


\title{Homework Assignment 01 \\
    \small Due: September 9, 2019}

\begin{document}
\maketitle

\section{The set of all strings containing the substring `11'}
\label{sec:s1}
(See \hyperref[fig:fsm1]{Figure 1})
\begin{figure}[ht]
\centering
\begin{tikzpicture}
% States %
\node[state, initial] at (-2,0) (S) {S};
\node[state] at (0,0) (1) {1};
\node[state, accepting] at (2,0) (11) {11};
% Edges %
\draw [->]
    (S) edge[above] node{1} (1)
    (S) edge[loop above] node{0} (S)
    (1) edge[above] node{1} (11)
    (1) edge[loop above] node{0} (1)
    (11) edge[loop right] node{0,1} (11);
\end{tikzpicture}
\caption{Finite Automata Representation for \hyperref[sec:s1]{Section 1}}
\label{fig:fsm1}
\end{figure}

\section{The set of all strings that start and end in the same symbol}
\label{sec:s2}
(See \hyperref[fig:fsm2]{Figure 2})
\begin{figure}[ht]
\centering
\begin{tikzpicture}
% States %
\node[state, initial, accepting] at (0,0) (S) {S};
\node[state] at (0,2) (0) {0};
\node[state] at (0,-2) (1) {1};
\node[state] at (-2,3) (00) {00};
\node[state] at (2,3) (01) {01};
\node[state] at (-2,-3) (10) {10};
\node[state] at (2,-3) (11) {11};
% Edges %
\draw [->]
    (11) edge[loop below] node{1} (11)
    (10) edge[loop below] node{0} (10)
    (01) edge[loop above] node{1} (01)
    (00) edge[loop above] node{0} (00)
    (S) edge[left] node{0} (0)
    (S) edge[right] node{1} (1)
    (0) edge[above] node{0} (00)
    (0) edge[below] node{1} (01)
    (1) edge[left] node{0} (10)
    (1) edge[right] node{1} (11)
    (0) edge[bend left, right] node{$\epsilon$} (S)
    (1) edge[bend left, left] node{$\epsilon$} (S)
    (00) edge[bend right, left] node{$\epsilon$} (S)
    (11) edge[bend right, right] node{$\epsilon$} (S);
\end{tikzpicture}
\caption{Finite Automata Representation for \hyperref[sec:s2]{Section 2}}
\label{fig:fsm2}
\end{figure}

\section{The set of all strings that do not start and end in the same symbol}
\label{sec:s3}
(See \hyperref[fig:fsm3]{Figure 3})
\begin{figure}[ht]
\centering
\begin{tikzpicture}
\node[state, initial] at (-2,0) (S) {S};
\node[state] at (-1,2) (0) {0};
\node[state] at (-1,-2) (1) {1};
\node[state] at (1,2) (01) {01};
\node[state] at (1,-2) (10) {10};
\node[state, accepting] at (2,0) (F) {final};
\draw [->]
    (S) edge[left] node{0} (0)
    (S) edge[left] node{1} (1)
    (0) edge[loop above] node{0} (0)
    (0) edge[above] node{1} (01)
    (1) edge[loop below] node{1} (0)
    (1) edge[above] node{0} (10)
    (01) edge[loop above] node{1} (01)
    (01) edge[bend left, below] node{0} (0)
    (01) edge[right] node{$\epsilon$} (F)
    (10) edge[loop below] node{0} (10)
    (10) edge[bend left, below] node{1} (1)
    (10) edge[right] node{$\epsilon$} (F);
\end{tikzpicture}
\caption{Finite Automata Representation for \hyperref[sec:s3]{Section 3}}
\label{fig:fsm3}
\end{figure}

\section{The set of all strings containing both of the substrings `01' and `10'}
\label{sec:s4}
(See \hyperref[fig:fsm4]{Figure 4})
\begin{figure}[ht]
\centering
\begin{tikzpicture}
% States %
\node[state, initial] at (-3,0) (S) {S};
\node[state] at (-1,1) (0) {0};
\node[state] at (-1,-1) (1) {1};
\node[state] at (1,1) (01) {01};
\node[state] at (1,-1) (10) {10};
\node[state, accepting] at (3,1) (010) {010};
\node[state, accepting] at (3,-1) (101) {101};
% Edges %
\draw [->]
    (S) edge[left] node{0} (0)
    (0) edge[above] node{1} (01)
    (01) edge[above] node{0} (010)
    (S) edge[left] node{1} (1)
    (1) edge[below] node{0} (10)
    (10) edge[below] node{1} (101)
    (0) edge[loop above] node{0} (0)
    (01) edge[loop above] node{1} (01)
    (010) edge[loop right] node{0,1} (010)
    (1) edge[loop below] node{1} (1)
    (10) edge[loop below] node{0} (10)
    (101) edge[loop right] node{0,1} (101);
\end{tikzpicture}
\caption{Finite Automata Representation for \hyperref[sec:s4]{Section 4}}
\label{fig:fsm4}
\end{figure}

\section{The set of all strings with an even number of 0's and an odd number of 1's}
\label{sec:s5}
(See \hyperref[fig:fsm5]{Figure 5} for a drawing representation)
\par
\textbf{Induction Hypothesis}:

\begin{figure}[ht]
\centering
\begin{tikzpicture}
\node[state, initial] at (-1,1) (A) {A};
\node[state] at (1,1) (B) {B};
\node[state, accepting] at (-1,-1) (C) {C};
\node[state] at (1,-1) (D) {D};
\draw [-]
    (A) edge[above] node{0} (B)
    (A) edge[left] node{1} (C)
    (B) edge[right] node{1} (D)
    (C) edge[below] node{0} (D);
\end{tikzpicture}
\caption{Given Finite Automata in \hyperref[sec:s5]{Section 5}}
\label{fig:fsm5}
\end{figure}

\end{document}
