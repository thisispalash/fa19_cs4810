\documentclass[11pt,letterpaper]{article}
\usepackage{preamble}

\title{Homework Assignment 04 \\
    \small Due: September 27, 2019}

\begin{document}
\maketitle

\section{(inverse) homomorphism}
\label{sec:1}
By the last question of the previous homework, 
$h^{-1}(L) = (a+ba)^*$ 
[Problem Statement: \textit{Let $L=(0+10)^*$, $h(a)=00$, and $h(b)=01$. What is $h^{-1}(L)$?}]. 
Passing $h^{-1}{L}$ through $h$ is trivial as $h(a)$ and $h(b)$ are known. Hence, $h(h^{\-1}{L}) = (h(a) + h(b)h(a))^* = (00+0100)^*$. \\
$L$ is described as $(0+10)^*$. Hence, the first part of the answer above ($(00)^*$) is contained within $L$. Moreover, the second part, $(0100)^*$ is contained within $L$ as well. \small{If $m=0$ and $n=10$, $L=(m+n)^*$. $00$ is the same as $(mm)^*$ and $(0100)^*$ is the same as $(mnm)^*$.} 
Hence, the result above, $h(h^{\-1}{L}) = (00+0100)^*$ is contained within $L = (0+10)^*$.

\section{which is larger?}
\begin{center}
    $h(h^{-1}(L)) \subseteq	L$
\end{center}
\paragraph{cheating}
The wording of the last question of the previous homework ``\textit{Show that your answer for $h(h^{-1}(L))$ is contained in $L$}" suggests the above relation.
\paragraph{informal}
From \hyperref[sec:1]{Section 1}, we see that $h(h^{\-1}{L}) = (00+0100)^*$ which is contained in $L=(0+10)^*$. However, $L$ contains many more strings which are not in $h(h^{\-1}{L})$ such as, $0, 10, 010, 0010, \cdots$ Based on this \textit{evidence}, $|L| > |h(h^{-1}(L))|$ and hence the above relation holds true.

\section{where are the proofs?!}
Let the theorem be denoted by $a$ and the proof be denoted by $\^{a}$. Then we can create a homomorphism such that all possible proofs are inserted randomly between the theorems. More formally, $h_1(a) = a \wedge h_1(\^{a}) = a$. Then $h^{-1}(book)$ gives the set of all strings where $\^{a}$ appears at random intervals, covering all possibilities (ie, a set containing $2^n$ strings). \\
Now, the required regular set is a set containing all the theorems with the proofs. Intersecting the book (passed through $h_1$) with the language of this regular set yields in all strings where the theorem is followed by the proof. Formally, $h_1^{-1}(book) \cap (a\^{a})^*$ is the resulting set. \\
Finally, let us create another homomorphism which maps all the theorems to itself and discards all the proofs. We can describe such a transformation by $h_2(a) = a \wedge h_2(\^{a}) = \epsilon$. Passing the set obtained above through this transformation, results in the set of all theorems but no proofs. \\
Therefore we can extract the set of all theorems by the following operation,
\begin{center}
    $h_2(h_1^{-1}(book) \cap (a\^{a})^*)$ where, \\
    $h_1(a) = a \wedge h_1(\^{a}) = a$, \\
    $h_2(a) = a \wedge h_2(\^{a}) = \epsilon$
\end{center}

\section{closure has come to me, myself}
\begin{center}
    $M = (Q,\Sigma,\delta,q_0,F)$
\end{center}
We can redirect the language to be passed under a described homomorphism $h$ and then that transformed set of strings to go through the machine $M$. We can finally also check if the machine accepts the initial string $x$ as well. On doing so, we construct a machine \hyperref[fig:4a]{as such}.
We can remove the second machine and redirect the output of $h$ in \hyperref[fig:4a]{Figure 1}. This allows us to remove the $and$ state as well. This construction looks like \hyperref[fig:4b]{Figure 2}.
Therefore an automaton that accepts $h(L(M))$ is the same as that which accepts $L(M)$. Hence, $M = (Q,\Sigma,\delta,q_0,F)$ describes a machine that accepts $h(L(M))$ \pagebreak
\begin{figure}[!htp]
\label{fig:4a}
\centering
\begin{tikzpicture}
% States %
\node[state, initial] at (-4,1) (S) {};
\node[state] at (-1,2) (h) {$h$};
\node[] at (0,0) (M) {$M$};
\node[] at (1,2) (Mh) {$M$};
\node[state] at (3,1) (A) {$and$};
\node[state, accepting] at (6,1) (F) {};
% Edges %
\draw [->]
    (S) edge[above] node{x} (h)
    (S) edge[above] node{x} (M)
    (h) edge[above] node{h(L)} (Mh)
    (M) edge[above] node{1} (A)
    (Mh) edge[above] node{1} (A)
    (A) edge[above] node{$1\wedge1$} (F);
\end{tikzpicture}
\caption{Closure under homormorphism}
\end{figure}
\begin{figure}[!htp]
\label{fig:4b}
\centering
\begin{tikzpicture}
% States %
\node[state, initial] at (-3,0) (S) {};
\node[state] at (0,2) (h) {$h$};
\node[] at (0,0) (M) {$M$};
\node[state, accepting] at (3,0) (F) {};
% Edges %
\draw [->]
    (S) edge[above] node{x} (h)
    (S) edge[above] node{x} (M)
    (h) edge[right] node{h(L)} (M)
    (M) edge[above] node{1} (F);
\end{tikzpicture}
\caption{Closure under homormorphism (simple)}
\end{figure}

\section{inverse substitution}


\end{document}