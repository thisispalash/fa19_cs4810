\documentclass[11pt,letterpaper]{article}
\usepackage{preamble}

\renewcommand\thesubsection{\thesection.\alph{subsection}}

\title{Homework Assignment 03 \\
    \small Due: September 20, 2019 }

\begin{document}
\maketitle

\section{Regex Construction}

\subsection{(a $\wedge$ b)}
\label{sec:1a}
The given requirement translates to the following automata.
\begin{figure}[!ht]
\label{fig:1a}
\centering
\begin{tikzpicture}
% States %
\node[state, initial] at (-2,0) (S) {S};
\node[state] at (0,1) (a) {a};
\node[state] at (0,-1) (b) {b};
\node[state, accepting] at (2,0) (F) {F};
% Edges %
\draw [->]
    (S) edge[loop above] node{c} (S)
    (S) edge[above] node{a} (a)
    (S) edge[below] node{b} (b)
    (a) edge[loop above] node{a,c} (a)
    (a) edge[above] node{b} (F)
    (b) edge[loop below] node{b,c} (b)
    (b) edge[below] node{a} (F)
    (F) edge[loop right] node{a,b,c} (F);
\end{tikzpicture}
\caption{Automata for \hyperref[sec:1a]{Section 1(a)}}
\end{figure}
\\The automata translates to the following regex:
\begin{center}
    $ c^*(a(a+c)^*b + b(b+c)^*a)(a+b+c)^* $
\end{center}

\subsection{$5^{th}$ from the end}
We know that the $5^{th}$ symbol from the end has to be a $1$. It can be followed by any combination of $0$'s and $1$'s, so long as there are only $4$. Also, it does not matter how many, if any, combinations of $0$'s and $1$'s precede the $5^{th}$ symbol from the end. Then, the required regex is,
\begin{center}
    $ (0 + 1)^*(1)(0+1)(0+1)(0+1)(0+1) $
\end{center}

\subsection{odd $1$'s}
The string must have at least one $1$ (the smallest odd number). That $1$ can be preceded by any number of $0$'s. Finally, it can be followed by any number of $0$'s or an even number of $1$'s. Therefore, the required expression is,
\begin{center}
    $ 0^*1(0 + 10^*1)^* $
\end{center}

\subsection{concatenation}
This string can be expressed by two expressions concatenated together. The first expression contains no $11$ and the second one contains no $00$. In the expression containing no pairs of $1$'s, every $1$ must be followed by a $0$. The string can end with a $0$ (ie, $\epsilon$ or a $1$. Hence, $ (0+10)^*(1+\epsilon) $ describes the first set. On flipping the $1$'s and $0$'s you can obtain the second set (of strings not containing $00$) to get $ (1+01)^*(0+\epsilon) $. Concatenating the two expression yields $ (0+10)^*(1+\epsilon)(1+01)^*(0+\epsilon) $ where $ (1+\epsilon)(1+01)^* $ can be simplified to just $ (1+01)^* $, giving us the final regex,
\begin{center}
    $(0+10)^* (1+01)^*(0+\epsilon)$
\end{center}

\pagebreak


\section{Machine Construction}
\label{sec:2}

\subsection{fools errand}
The regex may be constructed as an intersection of two sets, namely, the set that contains an odd number of $0$'s and the set containing the number of $1$'s divisible by $3$. The machines for both the sets can be represented as:
\begin{figure}[!ht]
\label{fig:2a1}
\centering
\begin{tikzpicture}
% States %
\node[state, initial] at (-1,0) (E) {E};
\node[state, accepting] at (1,0) (O) {O};
% Edges %
\draw [->]
    (E) edge[loop above] node{1} (E)
    (E) edge[bend left, above] node{0} (O)
    (O) edge[bend left, below] node{0} (E)
    (O) edge[loop above] node{1} (O);
\end{tikzpicture}
\caption{Odd number of $0$'s}
\end{figure}

\begin{figure}[!ht]
\label{fig:2a2}
\centering
\begin{tikzpicture}
% States %
\node[state, initial] at (-3,0) (S) {S};
\node[state] at (-1,0) (1) {1};
\node[state] at (1,0) (2) {2};
\node[state, accepting] at (3,0) (F) {F};
% Edges %
\draw [->]
    (S) edge[loop above] node{0} (S)
    (1) edge[loop above] node{0} (1)
    (2) edge[loop above] node{0} (2)
    (F) edge[loop above] node{0} (F)
    (S) edge[above] node{1} (1)
    (1) edge[above] node{1} (2)
    (2) edge[above] node{1} (F)
    (F) edge[bend left, below] node{1} (1);
\end{tikzpicture}
\caption{Number of $1$'s divisible by $3$}
\end{figure}
The machines give the regex \hyperref[fig:2a1]{$1^*0(1+01^*0)^*$} and \hyperref[fig:2a2]{$0^*10^*10^*1(0+10^*10^*1)^*$} respectively. However, having this information does not really help us as finding the intersection of the two is still quite difficult.
\pagebreak


\subsection{dfa}
The combination of the DFAs above results in the \hyperref[tab:2b]{state table} on the next page. Converting the state table to a \hyperref[fig:2b]{visual representation} is easy.
\begin{table}[!ht]
\label{tab:2b}
\centering
\begin{tabular}{c| c c}
     & $0$ & $1$  \\
    \hline
    $\rightarrow [E,S]$ & $[O,S]$ & $[E,1]$ \\
    $[O,S]$ & $[E,S]$ & $[O,1]$ \\
    $[E,1]$ & $[O,1]$ & $[E,2]$ \\
    $[O,1]$ & $[E,1]$ & $[O,2]$ \\
    $[E,2]$ & $[O,2]$ & $[E,F]$ \\
    $[O,2]$ & $[E,2]$ & $[O,F]$ \\
    $*[O,F]$ & $[E,F]$ & $[O,1]$ \\
    $[E,F]$ & $[O,F]$ & $[E,1]$ \\
\end{tabular}
\caption{State table for the combination of \hyperref[fig:2a1]{Figure 2} and \hyperref[fig:2a2]{Figure 3}}
\end{table}
\begin{figure}[!ht]
\label{fig:2b}
\centering
\begin{tikzpicture}
% States %
\node[state, initial] at (-6,4) (ES) {E,S};
\node[state] at (-3,4) (OS) {O,S};
\node[state] at (-6,0) (E1) {E,1};
\node[state] at (-3,0) (O1) {O,1};
\node[state] at (0,-2) (E2) {E,2};
\node[state] at (0,0) (O2) {O,2};
\node[state, accepting] at (3,0) (OF) {O,F};
\node[state] at (0,-4) (EF) {E,F};
% Edges %
\draw [->]
    (ES) edge[bend left, above] node{0} (OS)
    (ES) edge[left] node{1} (E1)
    (OS) edge[bend left, below] node{0} (ES)
    (OS) edge[right] node{1} (O1)
    (E1) edge[bend left, above] node{0} (O1)
    (E1) edge[bend right, below] node{1} (E2)
    (O1) edge[bend left, above] node{0} (E1)
    (O1) edge[above] node{1} (O2)
    (E2) edge[bend left, left] node{0} (O2)
    (E2) edge[left] node{1} (EF)
    (O2) edge[bend left, right] node{0} (E2)
    (O2) edge[above] node{1} (OF)
    (OF) edge[above] node{0} (EF)
    (OF) edge[bend right, above] node{1} (O1)
    (EF) edge[bend right, right] node{0} (OF)
    (EF) edge[bend left, left] node{1} (E1);
\end{tikzpicture}
\caption{Odd $0$'s and number of $1$'s divisible by $3$}
\end{figure}

\subsection{regex}
Using the State Removal Method (shown in class) on \hyperref[fig:2b]{Figure 4}, the \hyperref[fig:2c]{2 state automaton} is obtained.
\begin{figure}[!ht]
\label{fig:2c}
\centering
\begin{tikzpicture}
% States %
\node[state, initial] at (-2,0) (S) {S};
\node[state, accepting] at (2,0) (F) {A};
% Edges %
\draw [->]
    (S) edge[loop above] node{$w$} (S)
    (S) edge[bend left, above] node{$x$} (F)
    (F) edge[loop above] node{$y$} (F)
    (F) edge[bend left, below] node{$z$} (S);
\end{tikzpicture}
\caption{Final Automaton for \hyperref[sec:2]{required language}}
\end{figure}
\\
From the automaton, the regex created is, $(w + xy^*z)^*xy*$ \\
where, \\
$w = 00 + (01+10)(0+1(00)^*011)$ \\
$x = (01+10)(1(00)^*1+1(00)^*010)$ \\
$y = (1 + 01(00)^*(01+10))(1(00)^*1+1(00)^*010)$ \\
$z = (1 + 01(00)^*(01+10))(0+1(00)^*011)$
\paragraph{note}
We could try and simplify the above expression(s) but that would take too much time. Perhaps using a program might be better.

\section{Shuffle}
Let $M_A = (Q_A, \Sigma, \delta_A, q_A, F_A)$ represent the machine that accepts set $A$ and $M_B = (Q_B, \Sigma, \delta_B, q_B, F_B)$ represent the machine that accepts set $B$. An NFA can be created from the combinations of the two machines such that it moves, in the appropriate DFA on seeing the appropriate character. \\
The NFA can be described as $M = ((Q_A \times Q_B), \Sigma, \delta, (q_A \times q_B), (F_A \times F_B))$ where, $\delta ((s_A \times s_B),x) = \{(\delta_A (s_A,x) \times p_B), (p_A \times \delta_B (s_B,x))\}$ for all $x \in \Sigma$ [ie, if x changes state in A, then the state in B is not changed while the new state of A is made current and vice versa]. \\
The NFA accepts the string $w$ iff both, $M_A$ and $M_B$, are in the accepting states, where $w$ is the string $a_1b_1 \cdots a_nb_n$ where $a_1 \cdots a_n \in A$ and  $b_1 \cdots b_n \in B$ and every $a_i,b_i \in \Sigma^*$. This can be proved by structural induction,
\paragraph{Basis} Let $w = a_1b_1$, where $a_1,b_1 \in \Sigma^*$. We simply split the string into two parts $a_1$ and $b_1$. They will both independently reach their respective final states and hence $w$ will be accepted. It is trivial that the same holds if $w = b_1a_1$.
\paragraph{Induction} Any string $w = a_1b_1 \cdots a_nb_n$ can be reduced to the form $w = w_1 \cdots w_n$, where $w_i = a_ib_i$ and $a_i,b_i \in \Sigma^*$. This is the same as the basis multiple times and hence, $w$ will be accepted.
\paragraph{Conclusion} Hence, the class of regular sets is closed under shuffle (of two strings).
\pagebreak

\section{Reverse Homomorphism}
Given, $\Sigma = \{a,b\}$, $T = \{0,1\}$, $L = (0+10)^*$, and finally, $h(a)=00, h(b)=10$. By definition, $h^{-1}(L) = \{w \mid w \in \Sigma^* \wedge h(w) \in L\}$. Hence, the task is to simply create a regex that is contained in language $L$ after it passes through the homomorphism $h: \Sigma \to T$. \\
In language $L$, every $1$ is necessarily followed by a $0$. Hence, $w$ cannot end with a $b$. It can start with a $b$ however, as $01$ is contained in $L$. Notice that $a$ is possible anywhere as $00$ is always contained in $L$. Lastly, $\epsilon$ is also contained in $L$ (implying a $()^*$ on $w$). Therefore, $w$ may start with a $a$ or a $b$ but ends in a $a$. Also, multiple variations of $w$ may be concatenated.
\begin{center}
    Hence, $w = ((a+b)^*a^*)^*$ which can be simplified to $(a+ba)^*$.
\end{center}
\paragraph{note}
As a \textit{sanity check}, let us put $w$ through $h$;
$h(w)$ $=$ $(h(a)+h(b)h(a))^*$ $=$ $(00+0100)^*$ $\in$ $L$

\section{Lets delete!}
(If one thinks about some strings, they will realize that the reduced regex is of the form $(0^*10^*)^*$) \\
Let us consider a homomorphism described as $h_1(j) = j$ and $h_1(\jmath) = j$. The transformation $h_1^{-1}(L)$, where $L$ is the required language, would result in the set of $j_i$'s randomly interspersed with variable $\jmath_i$'s.
The intersection of this set with the set of languages represented by the regex $(\jmath+k)^*j(\jmath+k)^*$, we get the second set ($j$ is the character $1$ and $\jmath$ is the string $11$). \\
Now if we consider another homomorphism described as $h_2(j) = j$ and $h_2(\jmath) = \epsilon$. On passing the set obtained in the previous paragraph through this second homomorphism, we get the required set of strings described by $k^*jk^*$ which is the required set.
\\
\begin{center}
    $-$ x $-$
\end{center}

\end{document}