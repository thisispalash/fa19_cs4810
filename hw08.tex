\documentclass[11pt,letterpaper]{article}
\usepackage{preamble}

\title{Homework Assignment 08 \\
    \small Due: November 1, 2019}

\begin{document}
\maketitle

\section{one symbol CFL}


\section{not a CFL}
\begin{center}
    $L = \{a^i b^j c^k$ $|$ $i \neq j \wedge i<k\}$
\end{center}
Let us consider a string of the form $z = a^n b^{n+1} c^{n+1}, n \geq 0$. Let us further break up the string into $uvwxy$ with the constraint(s) $|vwx| \leq n \wedge |vx| > 0$. By the pumping lemma theorem, $L$ is a CFL iff $z \in L \wedge uv^iwx^iy \in L$ for some $i \geq 0$. Then, $vwx$ can be of five different forms -
\paragraph{case 1} $vwx = a^p, n \geq p \geq 1$ \\
In this case, we pump up and increase the number of $a$'s to unsatisfy the second condition ($i<k$). Hence, $uv^iwx^iy \notin L$
\paragraph{case 2} $vwx = a^pb^q, n \geq p+q \geq 1$ \\
Similar to case 1, we pump up and increase the number of $a$'s to unsatisfy the second condition ($i<k$). Note that we may even increase the number by just $1$ to unsatisfy $i \neq j$ as well. Hence, $uv^iwx^iy \notin L$
\paragraph{case 3} $vwx = b^p, n \geq p \geq 1$ \\
In this case, we pump down by 1 to match the number of $a$'s and $b$'s and falsify $i \neq k$. Hence, $uv^iwx^iy \notin L$
\paragraph{case 4} $vwx = b^pc^q, n \geq p+q \geq 1$ \\
Here we simply pump down to $0$ to make the number of $c$'s less than the number of $a$'s as there $x$ will contain some $c$'s, falsifying $i < k$. Hence, $uv^iwx^iy \notin L$
\paragraph{case 5} $vwx = c^p, n \geq p \geq 1$ \\
Similar to case 4, we pump down to $0$ to reduce the number of $c$'s to falsify $i<k$. Hence, $uv^iwx^iy \notin L$

\section{closure under complement}

\section{be useful or else..}
We check for two things for a variable of a CFG, $G = (V,T,P,S)$, to be useful: $(i)$ It is \textit{generating}; $(ii)$ It is \textit{reachable}. For a variable to be useful, it must be both of the above. Hence, we check for each condition and if a variable does not satisfy any condition, we eliminate it.
\paragraph{$X$ is generating} basically implies that the variable $X \in V$ reaches some terminal. More formally, if $X \Rightarrow^* w \in T^*$, then $X$ is generating.
\paragraph{$X$ is reachable} means that there is some derivation that reaches the variable $X \in V$ from the start symbol. More formally, if $X \Rightarrow^* \alpha X \beta$ for some $\alpha, \beta \in T^*$


\section{parse trees}
\begin{center}
\label{given:5}
    $S \rightarrow AB$ $|$ $BC$ \\
    $A \rightarrow BA$ $|$ $a$ \\
    $B \rightarrow CC$ $|$ $b$ \\
    $C \rightarrow AB$ $|$ $a$ \\
    String: $aabbab$
\end{center}
The string above has two parse trees from the \hyperref[given:5]{set of productions}. The two trees can be represented by the following derivations, \\
$S \Rightarrow AB \Rightarrow BAB \Rightarrow CCAB \Rightarrow aCAB \Rightarrow aABAB \Rightarrow aaBAB \Rightarrow aabAB \Rightarrow aabBAB \Rightarrow^* aabbab$ and $S \Rightarrow BC \Rightarrow CCC \Rightarrow^* aaC \Rightarrow aaAB \Rightarrow aaBAB \Rightarrow aabAB \Rightarrow aabBAB \Rightarrow^* aabbab$
\begin{table}[!ht]
    \label{tab:parse_tree}
    \centering
    \begin{tabular}{l c || c r}
    $S \Rightarrow AB \Rightarrow^* aabbab$ &
    &&
    $S \Rightarrow BC \Rightarrow^* aabbab$ \\
    \hline
    \Tree [.S [.A [.B [.C a ] 
                      [.C [.A a ]
                          [.B b ]]]
                  [.A [.B b ]
                      [.A a ]]]
              [.B b ]] &
    &&
    \Tree [.S [.B [.C a ] 
                  [.C a ]] 
              [.C [.A [.B b ]
                      [.A [.B b ]
                          [.A a ]]]
                  [.B b ]]] \\ 
    \end{tabular}
    \caption{Parse trees generated by the derivations}
\end{table}

\end{document}