\documentclass[11pt,letterpaper]{article}
\usepackage{preamble}

\title{Homework Assignment 08 \\
    \small Due: November 4, 2019}

\begin{document}
\maketitle

\section{CFL, but also regular}
\paragraph{informally} A language over any one symbol alphabet will contain strings ($z$, $|z| \geq 0$) that contain only one symbol (say, $a$). In this case, it is easy to see that a DFA can be constructed consisting of just two states that accepts all possible strings and only accepts the strings that contain the one symbol (in our case, $a$).
\begin{figure}[ht]
\label{fig:dfa}
\centering
\begin{tikzpicture}
% States %
\node[state, initial, accepting] at (-2,0) (S) {};
\node[state] at (1,0) (T) {trap};
% Edges %
\draw [->]
    (S) edge[loop above] node{$a$} (S)
    (S) edge[above] node{$a^{c}$} (T);
\end{tikzpicture} \\
where, $a^{c}$ is every symbol that is not $a$
\caption{DFA accepting all strings containing only the symbol $a$}
\end{figure}

\paragraph{formally} Let $L$ be a CFL over a single symbol, $a$. Additionally, let $L = L_1 \cup L_2$ where $|L_1| < n \wedge |L_2| \geq n$. $L_1$ is a then a regular set as it is a finite language. Consider a string $z \in L_2$, such that $z = uvwxy$. The pumping lemma states (for CFLs) that $L$ is regular iff $\forall z \in L, z = uvwxy \Rightarrow uv^iwx^iy \in L, \forall i \geq 0$. $L_2$ is given to be a CFL (it is also trivial to see that the lemma holds). As the language is described over a single symbol, the order of $uvwxy$ can be altered and we can Interchange the $v$ and $w$ in $z$. We can then define two more strings $s = uw$ and $t = vx$ which on substitution yields $\forall z \in L_2, z = sty \Rightarrow st^iy \in L_2, \forall i \geq 0$. By the pumping lemma (for regular sets), $L_2$ is a regular set as well. Hence, $L$ is a regular set.

\section{not a CFL}
\begin{center}
    $L = \{a^i b^j c^k$ $|$ $i \neq j \wedge i<k\}$
\end{center}
Let us consider a string of the form $z = a^n b^{n+1} c^{n+1}, n \geq 0$. Let us further break up the string into $uvwxy$ with the constraint(s) $|vwx| \leq n \wedge |vx| > 0$. By the pumping lemma theorem, $L$ is a CFL iff $z \in L \Rightarrow uv^iwx^iy \in L, \forall i \geq 0$. Then, $vwx$ can be of five different forms -
\paragraph{case 1} $vwx = a^p, n \geq p \geq 1$ \\
In this case, we pump up and increase the number of $a$'s to unsatisfy the second condition ($i<k$). Hence, $uv^iwx^iy \notin L$
\paragraph{case 2} $vwx = a^pb^q, n \geq p+q \geq 1$ \\
Similar to case 1, we pump up and increase the number of $a$'s to unsatisfy the second condition ($i<k$). Note that we may even increase the number by just $1$ to unsatisfy $i \neq j$ as well. Hence, $uv^iwx^iy \notin L$
\paragraph{case 3} $vwx = b^p, n \geq p \geq 1$ \\
In this case, we pump down by 1 to match the number of $a$'s and $b$'s and falsify $i \neq k$. Hence, $uv^iwx^iy \notin L$
\paragraph{case 4} $vwx = b^pc^q, n \geq p+q \geq 1$ \\
Here we simply pump down to $0$ to make the number of $c$'s less than the number of $a$'s as there $x$ will contain some $c$'s, falsifying $i < k$. Hence, $uv^iwx^iy \notin L$
\paragraph{case 5} $vwx = c^p, n \geq p \geq 1$ \\
Similar to case 4, we pump down to $0$ to reduce the number of $c$'s to falsify $i<k$. Hence, $uv^iwx^iy \notin L$

\section{CFL and a regular language}
A CFL, $L$ intersected with a regular language, $R$ is a CFL. We can construct a machine that runs the PDA (for $L$) and the DFA (for $R$) in parallel. The machine only accepts if and only if the PDA and the DFA accept.
\paragraph{Grammar Construction} 
\paragraph{case 1} Non empty $L$ without $\epsilon$ \\
In this case, $L$ has some CFG in Chomsky-Normal form such that every production is of the type $A \rightarrow BC$ or $A \rightarrow \sigma$ where $\sigma \in \Sigma$. We denote the set of states in the DFA that recognizes $R$ as $Q$. We can then construct a grammar $G' = (\Sigma,V',P',S')$ from $Q$ and the CFG for $L$, $G = (\Sigma,V,P,S)$, such that,
\begin{itemize}
    \item $V'$ is the set of triples of the form $[pAq]$, where $p,q \in Q \wedge A \in V$
    \item The productions in $P'$ are a combination of the productions in $P$ and for all $p,q,r \in Q$, 
    \begin{itemize}
        \item For each production of the form $A \rightarrow BC \in P$ we have $[pAr] \rightarrow [pBq][qCr] \in P'$
        \item For each production of the form $A \rightarrow \sigma \in P$ we have $[pAq] \rightarrow \sigma \in P'$
    \end{itemize}
    \item The start symbol in $G'$ is simply a triplet of the start symbol in $G$ and the start and accepting states in $Q$; ie, $S' = [sSf]$, where $s,f$ are start and accepting states in Q, respectively.
\end{itemize}
\paragraph{case 2} $\epsilon \in L$ \\
For the case where $\epsilon \in L$, we simply copy the production $S \rightarrow \epsilon$ to the new set of productions, allowing the new start symbol generate $\epsilon$ as well.

\section{be useful or else..}
We check for two things for a variable of a CFG, $G = (V,T,P,S)$, to be useful: $(i)$ It is \textit{generating}; $(ii)$ It is \textit{reachable}. For a variable to be useful, it must be both of the above. Hence, we check for each condition and if a variable does not satisfy any condition, we eliminate it.
\paragraph{$X$ is generating} basically implies that the variable $X \in V$ reaches some terminal. More formally, if $X \Rightarrow^* w \in T^*$, then $X$ is generating.
\paragraph{$X$ is reachable} means that there is some derivation that reaches the variable $X \in V$ from the start symbol. More formally, if $X \Rightarrow^* \alpha X \beta$ for some $\alpha, \beta \in T^*$
\paragraph{algorithm} We nullify all $\epsilon$-productions, substitute unit productions, and finally remove all useless by deleting all non generating and non reachable variables (in that order) and all productions involving such variables.

\section{parse trees}
\begin{center}
\label{given:5}
    $S \rightarrow AB$ $|$ $BC$ \\
    $A \rightarrow BA$ $|$ $a$ \\
    $B \rightarrow CC$ $|$ $b$ \\
    $C \rightarrow AB$ $|$ $a$ \\
    String: $aabbab$
\end{center}
The string above has two parse trees from the \hyperref[given:5]{set of productions}. The two trees can be represented by the following derivations, \\
$S \Rightarrow AB \Rightarrow BAB \Rightarrow CCAB \Rightarrow aCAB \Rightarrow aABAB \Rightarrow aaBAB \Rightarrow aabAB \Rightarrow aabBAB \Rightarrow^* aabbab$ and $S \Rightarrow BC \Rightarrow CCC \Rightarrow^* aaC \Rightarrow aaAB \Rightarrow aaBAB \Rightarrow aabAB \Rightarrow aabBAB \Rightarrow^* aabbab$
\begin{table}[!ht]
    \label{tab:parse_tree}
    \centering
    \begin{tabular}{l c || c r}
    $S \Rightarrow AB \Rightarrow^* aabbab$ &
    &&
    $S \Rightarrow BC \Rightarrow^* aabbab$ \\
    \hline
    \Tree [.S [.A [.B [.C a ] 
                      [.C [.A a ]
                          [.B b ]]]
                  [.A [.B b ]
                      [.A a ]]]
              [.B b ]] &
    &&
    \Tree [.S [.B [.C a ] 
                  [.C a ]] 
              [.C [.A [.B b ]
                      [.A [.B b ]
                          [.A a ]]]
                  [.B b ]]] \\ 
    \end{tabular}
    \caption{Parse trees generated by the derivations}
\end{table}
\paragraph{derivations} of the string $aabbab$ can be obtained by the use of dynamic programming on increasing lengths of the string. The table can be created as follows:
\begin{table}[]
    \centering
    \begin{tabular}{c | c c c c c c}
    $|z|$ & $z[0]$ & $z[1]$ & $z[2]$ & $z[3]$ & $z[4]$ & $z[5]$ \\
    \hline
    $6$ & $S_{0,5},S_{0,2}$ &
    \\
    $5$ & $A_{0,4}$ & $B_{1,2}$ &
    \\
    $4$ & $\O$ & $\O$ & $S_{2,5},C_{2,5}$ &
    \\
    $3$ & $B_{0,1}$ & $\O$ & $A_{2,3}$ & $S_{3,4},S_{3,5}$ &
    \\
    $2$ & $B_{0,1}$ & $S_{1,2},C_{1,2}$ & $\O$ & $S_{3,4},A_{3,4}$ &$S_{4,5},C_{4,5}$ &
    \\
    $1$ & $A_{0},C_{0}$ & $A_{1},C_{1}$ & $B_{2}$ & $B_{3}$ & $A_{4},C_{4}$ & $B_{5}$ \\
    \hline
    & $a$ & $a$ & $b$ & $b$ & $a$ & $b$ \\
    \end{tabular}
    \caption{Use of dynamic programming on increasing lengths of $aabbab$}
    \label{tab:dynamic}
\end{table}

\end{document}